\documentclass{article}

\usepackage{hyperref}
\usepackage{graphicx}
\DeclareGraphicsExtensions{.pdf,.png,.jpeg}
\usepackage{float}
\usepackage{xcolor}
\usepackage[brazil]{babel}
\usepackage[T1]{fontenc}
\usepackage[utf8]{inputenc}
\usepackage{indentfirst}
\title{Resumo 2 - Aproximando o princípio da prática da implementação de IA responsável   \\ \small Estudos em Inteligência Artificial}

\author{Anayran Pinheiro}

\begin{document}

\maketitle

\section{Resumo}

O artigo em questão tem como foco abordar como princípios e prática podem andar conjuntamente, começando ao abordar o uso da inteligência artificial (IA) na sociedade como um todo, seja na área social ou na área econômica. Em paralelo, mostra os esforços de grandes empresas que constantemente estão desenvolvendo e criando modelos de IA em criar regras éticas para manter os princípios focados em gerar um ganho, seja social ou econômico, sem quebrar a confiança necessária no sistema desenvolvido. Porém juntamente com a criação destes princípios e a aplicação prática, Schiff et al mostra que ainda há muito o que ser aproximado para que ambos conjuntos estejam trabalhando em maior sintonia um com o outro, já que enquanto um apresenta formas de balizar questões mais voltadas a sociedade, a prática visa o maior empenho e otimização computacional para entrega de resultados.

Isso se deve por conta de ainda exister uma grande complexidade entre ambos os campos, seja ainda no desenvolvimento de algoritmos puros ou nas questões de como estes algoritmos devem se comportar, visto que tais algoritmos não devem ter a potencialidade de ameaça aos usuários de tal sistema. Para defender este ponto, o autor aborda cinco temas que explanam a complexidade de aproximar a prática dos princípios responsáveis no uso de IA, mostrando como a complexidade dos impactos da IA podem ter um forte impacto na vida social e na ética, o custo destas consequências ser o mais dividido possível, as orientações de diferentes especialistas podem ser generalistas demais, que a existência de metodologias e ferramentas para a aplicação da IA com responsabilidade são mais difíceis do que se espera e que pessoas que são especialistas não costumam dividir o conhecimento com pessoas não tão especialistas visando alcançar um bem estar para a humanidade.

Diante tais desafios, o autor propõe uma espécie de critéria para um framework eficiente de implantação responsável de AI, onde este framework precisa ser amplo, operacional, flexível, iterativo, guiado e participativo, e defente que o framework que apresenta tais critérias tem uma forte chance de apresentar uma integração eficiente o suficiente para ser responsável eticamente. O autor também defende que o uso dos padrões definidos pela IEEE 7010 podem fornecer uma forte baliza para a aplicação dos conceitos apresentados anteriormente para um framework eticamente responsável, e a implantação por institutos sejam de ensino superior ou de organizações de negócios podem ser bons ambientes para ver tais sistemas em ação.

Como conclusão, o autor mostra um caso aplicado de reflorestamento auxiliado por IA, onde o sistema identifica se o crescimento da mata é orgânico ou se a destruição está sendo ocasionada por intervenção de queimadas, naturais ou mesmo interferência humana. E nisso ele mostra que a partir deste modelo de sistema, se tem como aplicar tanto as técnicas quanto as regras éticas em um sistema que vem para apresentar como essa integração é possível.


\begin{thebibliography}{9}
\bibitem{"Principles to Practices for Responsible AI: Closing the Gap".}
Schiff, Daniel
\textit{Retirado de Cornell University}
 https://arxiv.org/pdf/2006.04707.pdf, 09/09/2020
\end{thebibliography}








\end{document}
